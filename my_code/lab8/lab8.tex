\documentclass{article} %选择文档类型,我们如果是做期末大作业的话选article就可以了

\usepackage{anyfontsize}
%正如c++需要import库来实现各种各样的功能,Latex也需要调用宏包来实现各种各样的功能
\usepackage{amsmath}  %调用公式宏包
\usepackage{graphicx} %调用插图宏包
\graphicspath{{code_Latex/}}
\usepackage{ctex}     %调用中文宏包
\usepackage{float}
\usepackage{cite}


%\begin{document}这句话之前是导言区,这句话以后就开始写正文了
%可以把导言区理解为int main()函数之前的内容,而正文就是int main()主函数的部分了
\usepackage{geometry}
\geometry{left=1.5cm,right=0.5cm,top=1.0cm,bottom=1.5cm}

\begin{document}
    \title{\centerline{高性能计算实验报告}}
    \date{实验八}
    \author{信息学部 2023311704 王昕远}
    \maketitle
    \thispagestyle{empty}
\section{硬件配置}% \texttt{ OS版本: Linux wxy 5.15.153.1-microsoft-standard-WSL2 \#1 SMP Fri Mar 29 23:14:13 UTC 2024 x86\_64 x86\_64 x86\_64 GNU/Linux}
% \includegraphics[width=0.8\textwidth]{top.png}
CPU型号:AMD Ryzen 7 7745HX with Radeon Graphics 物理核数:8 频率:3600MHz\par

avx avx2 avx512\par
内存大小:Mem 7973212

\section{理论峰值计算}

\includegraphics[width=0.8\textwidth]{1.png}

\section{软件环境}
操作系统版本:Ubantu 22.04.3 LTS
MPI:version 4.0
BLAS:0.3.20


% \section{time-dgemm}
% \begin{tabular}{|c|c|c|c|c|}
%     \hline
%      & time \\ \hline
%     real & 0m38.037s \\ \hline
%     uesr  &1m15.746s \\ \hline
%     system &0m0.220s \\ \hline
%     (real+uesr)/system & 517.195s \\ \hline
% \end{tabular}

$$
$$




\end{document}